\section{Группы}

В этом параграфе мы сделаем огромный шаг в направлении абстрактной математики Это будет сложно и не понятно, но если вы это освоите, то всё остальное пойдёт уже в разы легче. Дать читателю заранее мотивацию зачем всё это нужно весьма трудно, скорее всего этот параграф оставит кашу в голове и к нему придётся возвращаться ещё ни один раз. Но как объяснить это так, чтобы было сразу понятно зачем это надо и осталось надолго в голове, я не знаю. Мне самому в своё время пришлось приложить усилия, чтобы освоиться с абстрактной алгеброй, и лёгкого пути, к сожалению, я не нашёл.

Первые практические применения изложенной в этом параграфе теории мы увидим уже в~\S4.5, так что осталось недолго, потерпите.

\begin{definition}
	Алгебраическая структура с единственной ассоциативной операцией называется \term{полугруппой}.
\end{definition}

\begin{definition}
	Если полугруппа имеет нейтральный элемент, то она называется \term{моноидом}.
\end{definition}

\begin{definition}
	\term{Группой} называется полугруппа, в которой есть нейтральный элемент и каждый элемент имеет обратный.
\end{definition}

\begin{definition}
	Если бинарная операция группы коммутативна, то группа называется \term{абелевой} (реже просто \term{коммутативной}).
\end{definition}

\begin{definition}
	Алгебраические подструктуры группы называются \term{подгруппами}. Подструктуры моноидов~--- \term{подмоноидами}.
\end{definition}

По-другому можно сказать, что подгруппа~--- это подмножество, само являющееся группой, а подмоноид~--- это подмножество моноида, являющееся моноидом. Приставка <<под>> очень часто встречается в математике именно в этом смысле: подкольцо~--- это подмножество кольца, являющееся само кольцом. Подполе~--- подмножество поля, являющееся полем. И так далее. В дальнейшем я не буду вводить отдельные определения для <<подсущностей>>, поскольку терминология это общая и понятия и так.

\begin{definition}
	\term{Порядком группы} называется мощность множества, на котором она определена. Обозначение то же, что и для множеств: $|G|$.
\end{definition}

\begin{definition}
	\term{Порядком элемента $x$} называется такое $n$, что $x^n = 1$. Обозначение $|x| = n$.
\end{definition}

Из всего сказанного нас будут главным образом интересовать группы. Абелевы группы кажутся очень удобными, но на практике с ними мы сталкиваться будем куда реже, чем с группами, операция которых некоммутативна. К тому же большинство абелевых групп, в отличие от неабелевых, очень просто устроено и мы это устройство изучим весьма быстро.

Бинарную операцию на группе будем пока записывать в мультипликативной форме, поскольку большинство примеров этого параграфа обычно используются в литературе именно в такой форме записи.

\begin{example}
	\term{Тривиальной} группой называется группа, состоящая из единственного элемента, являющегося нейтральным.
\end{example}

\begin{example}
	Множество $S_n$ перестановок $n$-элементного множества образует неабелеву группу относительно произведения перестановок. Эта группа называется \term{симметрической} (название происходит из геометрии, где перестановка вершин объектов определяет их симметрии; об этом мы будем говорить позже). Более общо множество биекций вида $X\to X$ на произвольном множестве $X$ (не обязательно конечном) так же образуют группу по операции композиции функций, которая обозначается как $S(X)$.
\end{example}

\begin{example}
	Множество чётных перестановок в группе $S_n$ образует её подгруппу. Эта группа обозначается как $A_n$ и называется \term{знакопеременной группой}.
\end{example}

\begin{example}
	По теореме \ref{thm:symgrpord} $|S_n| = n!$ и по упражнению \ref{ex:sgnpermgrp}, $|A_n| = \frac{n!}{2}$.
\end{example}

\begin{exercise}
	\term{Функцией Ландау} называется функция
	\[
	g(n) = \max_{x\in S_n} |x|
	\]
	Докажите, что
	\[
	g(n) = \max_{c_1+\ldots+c_n = n} \lcm\{c_1, c_2, \ldots, c_n\}
	\]
\end{exercise}

\begin{example}
	В таблице \ref{tb:v4} приведена \term{четвёртая группа Клейна} $V_4$.
\end{example}

\begin{table}[h]
	\centering
	\begin{tabular}{c|cccc}
		& $1$ & $a$ & $b$ & $c$ \\ 
		\hline $1$ & $1$ & $a$ & $b$ & $c$ \\ 
		$a$ & $a$ & $1$ & $c$ & $b$ \\ 
		$b$ & $b$ & $c$ & $1$ & $a$ \\ 
		$c$ & $c$ & $b$ & $a$ & $1$ \\ 
	\end{tabular}
	\caption{Четвёртая группа Клейна $V_4$}\label{tb:v4}
\end{table}

\begin{exercise}
	Докажите, что $V_4$ действительно задаёт группу. Абелева ли эта группа?
\end{exercise}

\begin{example}
	Таблицы \ref{tb:cmplxgrp} и \ref{tb:qtgrp}, если я нигде не ошибся при заполнении, определяют группы на множествах $\{1, -1, i, -i\}$ (группа комплексных единиц) и $\{1, -1, i, -i, j, -j, k, -k\}$ (группа кватернионных единиц) соответственно. Запоминать эти таблицы совершенно не нужно, эти группы мы рассмотрим далее отдельно в несколько другом контексте. Умножение комплексных единиц коммутативно, а кватернионных единиц~--- нет. Увидеть, что кватернионные единицы действительно образуют группу, пока весьма сложно (нужно перебрать $2\cdot 8^3$ выражений, чтобы убедиться в ассоциативности), однако ниже я покажу как это можно сделать без особых усилий.
\end{example}

\begin{table}[h]
	\centering
	\begin{tabular}{c|cccc}
		& $1$ & $-1$ & $i$ & $-i$ \\ 
		\hline $1$ & $1$ & $-1$ & $i$ & $-i$ \\ 
		$-1$ & $-1$ & $1$ & $-i$ & $i$ \\ 
		$i$ & $i$ & $-i$ & $-1$ & $1$ \\ 
		$-i$ & $-i$ & $i$ & $1$ & $-1$ \\ 
	\end{tabular} 
	\caption{Группа комплексных единиц}\label{tb:cmplxgrp}
\end{table}

\begin{table}[h]
	\centering
	\begin{tabular}{c|cccccccc}
		& $1$ & $-1$ & $i$ & $-i$ & $j$ & $-j$ & $k$ & $-k$ \\ 
		\hline $1$ & $1$ & $-1$ & $i$ & $-i$ & $j$ & $-j$ & $k$ & $-k$  \\ 
		$-1$ & $-1$ & $1$&$-i$ &$i$ & $-j$ & $j$ & $-k$ & $k$ \\ 
		$i$ & $i$ & $-i$ & $-1$ & $1$ & $k$& $-k$ & $-j$ & $j$ \\ 
		$-i$ & $-i$ & $i$ & $1$ & $-1$ & $-k$& $k$ & $j$ & $-j$ \\ 
		$j$ & $j$ & $-j$ & $-k$ &$k$ & $-1$& $1$ & $i$ & $-i$ \\ 
		$-j$ & $-j$ & $j$ & $k$ &$-k$ & $1$& $-1$ & $-i$ & $i$ \\ 
		$k$ & $k$ & $-k$ & $j$ & $-j$ &$-i$ & $i$  & $1$ &$-1$  \\ 
		$-k$ & $-k$ & $k$ & $-j$ & $j$ &$i$ & $-i$  & $-1$ &$1$  \\ 
	\end{tabular}
	\caption{Группа кватернионных единиц}\label{tb:qtgrp}
\end{table}

\begin{thm}
	В любой группе $(ab)^{-1}=b^{-1}a^{-1}$.
\end{thm}
\begin{proof}
	\[
	(ab)(b^{-1}a^{-1}) = a(bb^{-1})a^{-1} = aa^{-1} = 1
	\]
\end{proof}

\begin{exercise}
	Докажите, что если в группе для любого элемента верно $x^2 = 1$, то эта группа коммутативная.
\end{exercise}

Обратите внимание, что формулировка теоремы, как и её доказательство, полностью дублируют теорему~\ref{thm:perminv} для перестановок. Таким образом мы ту теорему смогли обобщить на случай произвольных групп. Ещё один плюс абстракции, хоть и опять довольно слабенький.

\begin{exercise}
	Докажите, что множество обратимых элементов моноида образует группу.
\end{exercise}

\begin{exercise}\label{ex:grpint}
	Даны группы $G$ и $H$, пересекающиеся как множества. Более того, на пересечении $G\cap H$ их операции совпадают. Докажите, что $G\cap H$~--- группа. Докажите, что это же верно для пересечения произвольного семейства групп, в том числе бесконечного.
\end{exercise}

\begin{example}
	Пусть $\Sigma$~--- произвольное множество, элементы которого мы назовём генераторами моноида. Элементами моноида будем считать конечные последовательности элементов из $\Sigma$, включая пустую последовательность, которую для удобства обозначим как 1. Бинарной операцией на этом моноиде мы зададим операцию конкатенации (то есть <<склеивания>> последовательностей: $abc\cdot de = abcde$). Моноид, построенный таким образом, называется \term{свободным}. Слово <<свободный>> в данном случае имеет смысл <<без ограничений>>. В силу ассоциативности конкатенации мы действительно получаем моноид.
\end{example}

\begin{example}
	Пример умножения в моноиде с генераторами $\Sigma=\{a, b\}$:
	\[
	abbba\cdot ab = abbbaab
	\]
\end{example}

Для удобства несколько идущих подряд символов можно записывать, указав просто число повторений сверху:
\[
abbbaab = ab^3a^2b
\]

\begin{example}
	По аналогии со свободным моноидом можно построить \term{свободную группу}. Конструкция отличается тем, что для каждого $x\in\Sigma$ мы добавляем так же \term{противоположный} элемент $x^{-1}$, связанный с $x$ соотношением $xx^{-1}=x^{-1}x=1$. Обозначается такая группа как $F_\Sigma$. Пример ниже показывает умножение в группе $F_{\{a, b\}}$:
	\[
	ab^{-3}a^2\cdot a^{-2}b^2ab = ab^{-1}ab
	\]
\end{example}

Если в свободную группу помимо сокращения $xx^{-1}=1$ добавить ещё какие-то сокращения, то полученное множество так же будет группой. Если $\Sigma = \{a, b, c\}$, а $w, t$~--- слова этого алфавита, то группа, полученная путём <<сокращения>> этих слов в $F_\Sigma$ обозначается как $<a, b, c | w = t = 1>$.

\begin{exercise}
	Докажите, что таким способом действительно определяется группа. Покажите, что эта группа неабелева.
\end{exercise}

\begin{example}
	Группа $<g|g^n = 1>$ называется \term{циклической группой порядка $n$} и обозначается как $Z_n$.
\end{example}

\begin{example}
	Рассмотрим $Z_{30}$. В этой группе $g^{15}g^{20} = g^5$.
\end{example}

\begin{example}
	Легко увидеть, что группа комплексных единиц изоморфна $Z_4$:
	\[
	h(1) = 1, h(g) = i, h(g^2) = -1, h(g^3) = -i
	\]
\end{example}

\begin{exercise}
	Докажите, что
	\[
	V_4 = <a, b | a^2 = b^2 = (ab)^2 = 1>
	\]
\end{exercise}

\begin{definition}
	Пусть $G$ и $H$~--- группы. \term{Прямым произведением групп} $G\times H$ называется их прямое произведение как множеств с заданной бинарной операцией
	\[
	(a, b) \cdot (c, d) = (ac, bd)
	\]
\end{definition}

\begin{exercise}
	Докажите, что прямое произведение групп является группой.
\end{exercise}

\begin{example}
	Легко увидеть, что $V_4 \cong Z_2 \times Z_2$. Изоморфизм определяется как
	\[
	h(a) = (x, 1), h(b) = (1, x), h(c) = (x, x)
	\]
\end{example}

Вместо того, чтобы указывать какие именно последовательности букв <<сокращаются>>, можно задавать произвольные замены. Пусть, например, мы решили, что всегда можно заменять слово $w$ на $v$, то есть что в нашей группе $w=v$. Но это значит, что $wv^{-1} = 1$, то есть любая замена может быть сведена к сокращению символов и наоборот. Чуть более формально мы это рассмотрим в последующих параграфах.

\begin{exercise}
	Покажите, что группа кватернионных единиц может быть определена как
	\[
	<-1, i, j, k | i^2 = j^2 = k^2 = ijk = -1, (-1)^2 = 1>
	\]
	Из этого определения легко построить таблицу, приведённую выше, что автоматически доказывает, что она действительно задаёт группу.
\end{exercise}

\begin{thm}
	Пусть $A\subset G$~--- произвольное подмножество группы $G$. Тогда однозначно определена подгруппа $G$, наименьшая по размеру, содержащая $A$ как подмножество.
\end{thm}
\begin{proof}
	Рассмотрим семейство всех подгрупп $G$, содержащих $A$. По результату упражнения \ref{ex:grpint} пересечение всех групп этого семейства будет снова группой. Очевидно, что это наименьшая подгруппа $G$, содержащая $A$.
\end{proof}

Если $A\subset G$, то минимальная подгруппа $G$, содержащая $A$, обозначается как $<A>$. Говорят, что множество $A$ \term{порождает} подгруппу $<A>$.

\begin{thm}
	Пусть $G$~--- произвольная группа, и $x$~--- её элемент. Тогда подгруппа $<x>$ будет либо свободной группой одного элемента, либо циклической.
\end{thm}
\begin{proof}
	Начнём строить последовательность $x, x^2, x^3, \ldots$ Здесь есть два варианта: либо элементы последовательности всегда будут различны, либо встретятся повторяющиеся.
	
	Если найдутся такие числа $k<n$, что $x^k = x^n$, то умножив это равенство с обоих сторон на $x^{-k}$, получим $1 = x^{n-k}$. Это значит, что элементы нашей последовательности будут повторяться циклически:
	\[
	x, x^2, x^3, x^4, \ldots, x^{n-k-1}, 1, x, x^2, x^3, \ldots
	\]
	Произведение любых двух элементов этой последовательности даёт так же элемент этой последовательности:
	\[
	x^a x^b = x^{a + b} = x^{(a+b)\Mod (n-k)}
	\]
	Здесь $\Mod$ обозначает остаток от деления. Обратным элементом для $x^a$ будет являться $x^{a+n-k}$. Таким образом элементы $1, x, x^2, \ldots, x^{n-k-1}$ образуют циклическую группу. Очевидно, что это наименьшая подгруппа $G$, содержащая $x$.
	
	Если же элементы последовательности $x, x^2, \ldots$ все будут различны, то подгруппа $<x>$ будет так же содержать обратные элементы $x^{-1}, x^{-2}, \ldots$ и нейтральный элемент 1. Но это в точности свободная группа $F_{\{x\}}$.
\end{proof}

\begin{corollary}
	Любая конечная группа содержит в себе циклическую подгруппу.
\end{corollary}

\begin{definition}
	Если существует набор элементов $g_i \in G$ таких, что $<\{g_i\}> = G$, то эти элементы называются \term{генераторами} группы $G$ и говорят, что группа ими \term{порождается}.
\end{definition}

\begin{example}
	Элемент $i$ является генератором группы комплексных единиц, элементы $i, j$ являются генераторами группы кватернионных единиц.
\end{example}

\begin{exercise}
	Докажите, что не существует эпиморфизмов из группы кватернионных единиц на группу комплексных единиц.
\end{exercise}

В этом упражнении поможет наблюдение, что любой гомоморфизм целиком определяется своими значениями на генераторах, при этом правда надо ещё доказать, что конкретное отображение генераторов не приводит к противоречию при определении гомоморфизму. Допустим, мы решили, что гомоморфизм $h$ отображает кватернионные единицы $i$ и $j$ в комплексную $i$. Тогда
\[
h(-1) = h(i^2) = h(i)^2 = i^2 = -1
\]
В то же время
\[
h(-1) = h(k^2) = h(ijij) = i^4 = 1
\]
Полученное противоречие показывает, что никакой гомоморфизм не может отображать одновременно и $i$ и $j$ в $i$.

\begin{thm}
	Гомоморфизмы отображают нейтральный элемент в нейтральный и обратные элементы в обратные.
\end{thm}
\begin{proof}
	\[
	h(x) = h(1x) = h(1)h(x)
	\]
	Таким образом видно, что $h(1)$ является нейтральным элементом для $h(x)$. Аналогично доказывается и то что $h(x^{-1})=h(x)^{-1}$:
	\[
	h(1) = h(xx^{-1}) = h(x)h(x^{-1})
	\]
\end{proof}

\begin{definition}
	\term{Ядром} гомоморфизма групп называется множество $\{x\in G|h(x) = 1\}$.
\end{definition}

\begin{exercise}
	Докажите, что ядро гомоморфизма ${h:G\to H}$ является подгруппой $G$. (И вспомните, что образ $h(G)$ является подгруппой $H$).
\end{exercise}

Кстати, для любой нетривиальной группы существует по крайней мере два эндоморфизма: эндоморфизм $h(x)=1$ для любого $x$ и автоморфизм $h(x)=x$. Они однако настолько тривиальны, что о них обычно не говорят.

\begin{exercise}
	Докажите, что отношения <<существует эпиморфизм>> и <<существует мономорфизм>> задают предпорядок на заданном семействе групп.
\end{exercise}

\begin{exercise}
	Постройте эпиморфизм $h:S_n\to Z_2$.
\end{exercise}

\begin{exercise}
	Докажите, что $Z_6\cong Z_2\times Z_3$.
\end{exercise}

\begin{definition}
	$G$ называется \term{группой действий над множеством M}, если существует мономорфизм $h:G\to S(M)$.
\end{definition}

\begin{definition}
	Пусть $M_n$ --- $n$-угольник с равными сторонами и углами (будем пока опираться на интуитивные образы, не ударяясь в строгие определения). Рассмотрим его отражения относительно центральных осей а так же повороты, переводящие его вершины в его же вершины. Множество поворотов и отражений такого $n$-угольника называется \term{группой диэдра} $D_{2n}$. Пример поворотов и отражений группы $D_{10}$ приведён на рисунке.
\end{definition}

\begin{tikzpicture}

\node [above] at (90:2) {a};
\node [right] at (19:2) {b};
\node [below] at (306:2) {c};
\node [below] at (234:2) {d};
\node [left] at (162:2) {e};

\end{tikzpicture}

Группа действий таким образом~--- это некоторое подмножество преобразований множества $M$. Если $M$, элементам которой соответствуют некоторые биективные преобразования множества $M$. Интересно, что любая группа является группой действий. Если разглядывать таблицы \ref{tb:v4}, \ref{tb:cmplxgrp} и \ref{tb:qtgrp}, то можно заметить, что каждая строка и каждый столбец этих таблиц~--- это некоторая перестановка элементов группы. Осталось формализовать и доказать это соображение.

\begin{thm}\label{thm:keli}
	Любая группа $G$ является группой действий над $G$ как над множеством. В частности, любая конечная группа порядка $n$ вкладывается в $S_n$.
\end{thm}
\begin{proof}
	Достаточно показать, что для любого $g\in G$ отображение $x\mapsto gx$ является биекцией. Предположим, что всё таки это не биекция и что различные элементы $x_1$ и $x_2$ отображаются в один элемент: $gx_1 = gx_2$. Однако, есть теперь умножить это равенство слева на $g^{-1}$, получим $x_1 = x_2$. Полученное противоречие доказывает теорему.
\end{proof}

Эта теорема не очень полезна на практике. Группа кватернионных единиц имеет порядок 8, но в то же время теорема \ref{thm:keli} лишь позволяет утверждать, что она вкладывается в группу $S_8$, порядок которой $8!=40320$. То есть какой-то реальной информации мы получили мало. В то же время само представление о том, что каждая группа~--- это подмножество какой-то группы перестановок, может помогать интуиции. По крайней мере теперь группы уже не должны выглядеть совсем уж абстрактными~--- мы всегда знаем, что их элементы могут быть интерпретированы как некоторые преобразования множеств. В худшем случае, как преобразования самого множества, на котором группа задаётся.

Мы могли бы рассматривать не группы действий, а моноиды действий, и иногда это полезно. Представим себе, что некоторый объект может находиться в каждый момент времени в каком-то одном состоянии. Множество его состояний~--- это множество $M$. Каждое конкретное действие переводит этот объект из одного состояния в другое. Простейший пример~--- компьютерная программа. При нажатии на клавиши программа переходит из одного состояния в другое. Все возможные состояния программы образуют множество $M$. Клавиши компьютера образуют алфавит $\Sigma$. Множество всех последовательностей нажатия на клавиши обозначим как $A$ (это фактически слова, составленные из символов алфавита $\Sigma$). Будем считать, что две последовательности из $A$ эквивалентны, если они всегда имеют одинаковый эффект в программе. Если мы факторизуем $A$ по этому отношению, то получим в точности множество всех возможных действий на $M$. Это и есть наш моноид действий.

В данном контексте разница между моноидом и группой такая, что если мы имеем группу действий, то любое действие может быть обратимо, так как в группе всегда есть обратный элемент. Если же наши действия в программе образуют лишь моноид, то может существовать такая последовательность нажатия на клавиши, что программа уже никогда не сможет вернуться в исходное состояние.

Именно моноиды действий лежат в основе концепции \term{конечных автоматов}, которые очень популярны в информатике. Конечный автомат~--- это абстракция примитивной машины, которая имеет конечное число внутренних состояний, которые по известному правилу могут меняться при поступлении в автомат символов некоторого алфавита $\Sigma$. В информатике конечные автоматы рассматривают как какое-то самостоятельное понятие и определяют через функцию $\Sigma\times M\to M$ изменения состояний при вводе символов. В действительности конечные автоматы~--- это всего лишь частный случай моноида действий. В качестве моноида тут выступают слова над $\Sigma$, в качестве множества $M$~--- состояния автомата.

Если у автомата задать начальное и конечное состояние, то можно говорить о том, что некоторые слова приводят автомат в конечное положение, а некоторые нет. Таким образом с помощью автоматов можно определять множества слов, то есть языки (я не буду вдаваться в подробности, но такое описание языков менее выразительно, нежели контекстно-свободные грамматики).

Следующие два упражнения не обязательны.

\begin{exercise}
	Пусть конечный автомат имеет $n$ состояний и он принимает некоторое слово длинной $n$ или больше. Докажите, что в этом случае в этом слове есть такой участок, который можно либо вообще выкинуть, либо повторить произвольное число раз, и автомат по-прежнему будет принимать это слово. В частности, если автомат принимает хотя бы одно слово длиннее $n-1$, то язык, который он принимает, бесконечен. (Подсказка: нарисуйте состояния автомата и переходы между ними в виде графа).
\end{exercise}

\begin{exercise}
	Пусть автомат принимает на вход десятичные цифры, а его состояния~--- это числа от 0 до наперёд заданного $k$. Если автомат находится в состоянии $s$, то принимая на вход цифру $x$, он переходит в состояние $10s + x \Mod k$. Начальным и конечным автоматом является состояние 0. Докажите, что этот автомат принимает на вход в точности числа, делящиеся на $k$.
\end{exercise}

Эти упражнения показывают интересное явление: если число $w$ делится на $k$ и имеет по крайней мере $n>k$ цифр, то число $10^{k-1}w + w$ так же делится на $k$.